\documentclass[a4paper]{article}
\usepackage[14pt]{extsizes}
%%% Работа с русским языком
\usepackage{cmap}					% поиск в PDF
\usepackage{mathtext} 				% русские буквы в фомулах
\usepackage[T2A]{fontenc}			% кодировка
\usepackage[utf8]{inputenc}			% кодировка исходного текста
\usepackage[english,russian]{babel}	% локализация и переносы

%%% Дополнительная работа с математикой
\usepackage{amsfonts,amssymb,amsthm,mathtools} % AMS
\usepackage{amsmath}
\usepackage{icomma} % "Умная" запятая: $0,2$ --- число, $0, 2$ --- перечисление

%% Номера формул
%\mathtoolsset{showonlyrefs=true} % Показывать номера только у тех формул, на которые есть \eqref{} в тексте.

%% Шрифты
\usepackage{euscript}	 % Шрифт Евклид
\usepackage{mathrsfs} % Красивый матшрифт

%% Свои команды
\DeclareMathOperator{\sgn}{\mathop{sgn}}

%% Перенос знаков в формулах (по Львовскому)
\newcommand*{\hm}[1]{#1\nobreak\discretionary{}
{\hbox{$\mathsurround=0pt #1$}}{}}

%%% Работа с картинками
\usepackage{float}
\usepackage{graphicx}  % Для вставки рисунков
\graphicspath{{pictures/}}  % папки с картинками
\setlength\fboxsep{3pt} % Отступ рамки \fbox{} от рисунка
\setlength\fboxrule{1pt} % Толщина линий рамки \fbox{}
\usepackage{wrapfig} % Обтекание рисунков и таблиц текстом

%%% Работа с таблицами
\usepackage{array,tabularx,tabulary,booktabs} % Дополнительная работа с таблицами
\usepackage{longtable}  % Длинные таблицы
\usepackage{multirow} % Слияние строк в таблице

\usepackage{cite}
\usepackage{csquotes}


%%% Заголовок
\author{Катнов Артем}
\title{Курсовая работа по теме: \\ Численное моделирование динамики частиц дроби в рудоразмольной мельнице методом дискретных элементов}
\date{\today}

\usepackage[left=3cm,right=1.5cm,top=2cm,bottom=2cm]{geometry}
\linespread{1.5}
%\parindent=1.25cm
%\oddsidemargin=4.6mm
%\textwidth=16cm
%\headheight=0cm
%\headsep=0cm
%\topmargin=-1cm
%\textheight=25.7cm

%\usepackage{caption}
%\usepackage{flafter}
%\usepackage{footmisc}

\usepackage{xcolor}
\usepackage{hyperref}
% цвета для гиперссылок
\definecolor{linkcolor}{HTML}{0000FF} % цвет ссылок
\definecolor{urlcolor}{HTML}{0000FF} % цвет гиперссылок
\hypersetup{linkcolor=linkcolor, urlcolor=urlcolor, colorlinks=true}

\usepackage{cleveref}
%\usepackage{underscore}
%\usepackage{etoolbox}
%\usepackage{lastpage}
%\usepackage{titlesec}
%\usepackage{flafter}
%\usepackage{color}
%\usepackage{mfirstuc}
%\usepackage{nomencl}
%\usepackage{iftex}

\begin{document}

\section{Титульник}

Добрый вечер уважаемая комиссия, меня зовут Артем Катнов, мой научный руководитель Жуков Никита Александрович, тема моей исследовательской работы "Численное моделирование динамики частиц в рудоразмольной мельнице методом дискретных элементов".

\section{Содержание}

В данной презентации я бы хотел рассказать об идее метода дискретных элементов, описать построенную математическую модель и обсудить применение данного метода к конкретной задаче рудоразмольной шаровой мельницы.

\section{Дискретная среда}

Метод дискретного элемента -- это семейство численных методов предназначенных для расчёта движения большого количества частиц.
Метод был первоначально применён Cundall в 1971 для решения задач механики горных пород.

\section{Цель работы}

В данной работе будет исследоваться шаровая рудоразмольная мельница.
Цель: применения метода дискретного элемента для исследования динамики частиц дробии руды во вращающемся барабане рудоразмольной мельницы.

\section{Алгоритм в общем виде}

Моделирование динамики системы происходит во времени.

В общем виде алгоритм метода представляет собой цикл в котором происходит сначала расчет всех сил действущих на каждую из частиц, а потом обновление положения частиц.
В ходе расчета сил также проводится итерационная процедура (в виду нелинейности задачи).
В ходе же обновления частиц также их количество может поменяться (если какая-то из частиц руды будет разрушена).

\section{Алгоритм в частном виде}

Данная модель предполагает наличие всего трех степеней свободы.

На слайде продемонстрированы три возможных состояния частицы (возможны также различные комбинации первых двух):

--- контактное взаимодействие частица-частица;

--- контактное взаимодействие частица-шар;

--- частица в свободном полете.

\section{Алгоритм еще раз}

1) Определение контактов

2) Расчет контактных сил

3) Определения кинематического закона

4) Определение нового положения шаров
%На каждом шаге при нахождении пересекающихся шаров для них происходит переход в локальную систему координат и расчёт всех возникающих сил, потом перевод всех найденных значений в глобальную систему.
%В зависимости от сил, действующих на частицу, происходит выбор закона движения на данном шаге.
%При наличии контактных сил проводится итерационная процедура ввиду нелинейности решаемой задачи.


\section{Контактные силы}
На данном слайде представлены все контактные силы действующие на шары в момент взаимодействия.
Это нормальная отталкивающая сила, сила трения скольжения и моменты трения качения и скольжения.
Подобное моделирование силовых факторов от трения описано в статье, представленной на слайде и довольно распространено при построении МДЭ.

\section{Контактные силы в нормальном направлении}

Контактные силы находятся по контактной модели Герца-Миндлина.
Таким образом жесткость считается по представленной на слайде формуле 2 и сила зависит от вхождения нелинейно.

\section{Контактные силы в тангенциальном и окружном направлениях}

На данном слайде представлены формулы по которым (в рамках задачи Герца) находятся силовые факторы в тангенциальном и окружном направлениях. 
Соответственно они зависят от направления относительных скоростей, значения соответствующих коэффициентов трения и нормальной силы.


\section{Силы диссипации}

Взаимодействия двух шаров можно представить в виде пружины и демпфера.
Находятся силы диссипации тоже по контактной модели Герца-Миндлина и рассчитываются в нормальном и тангенциальном направлениях.

\section{Кинематика}

Думаю важно сделать оговорку, что в этой работе происходит расчёт для трёх степеней свободы.
Задача которую решает данный метод изначально нелинейна, потому что силы зависят от перемещения, а перемещения зависят от сил. 
В качестве упрощения мы принимаем, что силы на  малом шаге изменяются линейно.
Скорость изменения сил называется рывок.
Его мы находим в ходе итерационного уточнения.
На слайде он обозначен переменной $b$ с соответствующим индексом.

Итерационное уточнение представляет собой цикл, целью которого является нахождение ускорения в следующий момент времени.
После нахождения данного ускорения мы и определяем рывок по формулам, представленным на слайде.
Расчёт рывка происходит по каждому их трёх стрепеней свободы.

После нахождения рывков и ускорений в локальной системе координат (в нормальном и тангенциальном направлениях) происходит переход в глобальную систему координат через матрицу перехода.

\section{Алгоритм итерационного уточнения}

На данном слайде представлен алгоритм итерационного уточнения.
В ходе этого цикла мы предполагаем следующее положение шара (исходя из его скорости и ускорения от данного взаимодействия) и по закону расчёта контактных сил находим обновленную силу и соответственно ускорение.
Расчет продолжается пока изменение ускорения в ходе итерационного процесса нельзя будет считать достаточно малым.

\section{Совокупность уравнений}

На данном слайде представлены совокупности уровнений с контактным взаимодействием и без него.
В обоих случаях мы записываем уравнение равновесия, из которого находим ускорение рывок и определяем положение шара в следующий момент времени.

\section{Модель разрушения}

Для расчета используется модель замещения баланса популяции (PBRM) - частицы заменяются набором дочерних фрагментов в момент разрушения. 
Каждый шар хранит накопленную энергию разрушения.
Определяется она по формуле 9.
Добавок энергии Е определяется по формуле 10.
Далее происходит расчет вероятности разлома частицы и определение того сломается она или нет с заданной вероятностью.

\section{Модель разрушения: проблема}

Далее, если частица разрушена, то с помощью нормального распределения случайным образом определяется количество новых частиц, образованных на месте старой.
Расчет размера и положения новых частиц происходит из равенства масс и кинетической энергии.


\section{Рудоразмольная мельница}

Вся эта модель построена для расчета различных режимов работы шаровой мельницы.
На слайде представлено в общем виде что она из себя представляет.
Через одну из цапф происходит загрузка материала, внутри барабана происходит перемалывание, и через другую цапфу происходит выгрузка материала, который может пройти через сито.
Для обеспечения подобного эффекта в данной работе периодически происходит добавление новой частицы руды в предположительно наименее забитую шарами часть мельницы и также удаление частиц руды, проходящих через сито.
Сито в свою очередь конечно имеет некоторую пропускную способность и туда проходят только достаточно маленькие частицы руды.
Расположено оно по центру аналогично рисунку.


\section{Реальные параметры}

На данном слайде представлены реальные параметры величин, учавствующих в расчетах.
Здесь представлены величины имеющие реальные параметры, а на следующем слайде величины полученные эмпирически и в целом необходимые в доработке для конкретных значений.
  


\section{Результаты}

\subsection{Каскадный режим}

Синие шары -- дробь

Коричневые -- руда 

Измением скорость вращения мельницы

характеризуется перекатыванием шаров по другим слоям

Наблюдаем выход на установившийся процесс

\subsection{Смешанный режим}

является переходным

\subsection{Водопадный режим}

характеризуется отрывом от поверхности мельницы и дальнейшим полетом по параболе

Увеличение энергии объяняется увеличением скорости вращения мельницы и соттетствено большей энергией приходящей системе извне

\subsection{Закритический режим}

При достижении определенной скорости шары не отрываются от поверхности барабана и это приводит к картине которую вы видите на слайде.

\subsection{График распределения эффективности работы мельницы}

На графике представлена зависимость эффективности перемалывания руды от скорости вращения шаровой мельницы. 
Эффективность определяется как отношение массы просеянной руды к суммарной массе руды.
Как видно из этого графика при увеличении скорости растет и эффективность работы мельницы.
Стоит отметить, что для всех режимов работы выбрана одинаковая скорость подачи руды.
В какой-то момент выбранная скорость подачи материала в цапфу становится недостаточна.

Также здесь явно прослеживается выход графика на горизонтальную асимптоту -- скорость приращения эффективности после точки 7.5 рад / с падает. 
Таким образом дальнейшее увеличение скорости вращения мельницы не приводит к значительному увеличению эффективности ее работы. 
В то же время -- на практике же используют каскадно-водопадный и водопадный режимы работы.
Это связано с прочностными характеристиками мельницы и энергетическими затратами по содержанию и обслуживанию мельницы.
Таким образом результаты работы модели подтверждат, что выбор каскадно-водопадного режима является рациональным с точки зрения эффективности мельницы и накладываемых прочностных и экономических ограничений

Рост эффективности (вплоть до 1) при больших скоростях вращения барабана связан с тем, что скорость разрушения и вывода руды из полости барабана превышает скорость подачи.
Логично предположить, что при увеличении скорости вращения мельницы также должна увеличиваться частота подачи материала.


\section{Вывод}

На этом у меня все.

В данной работе:

	--- построена математическая модель взаимодействия частиц;

	--- рассмотрены 4 характерных режима работа мельницы: каскадный, каскадно-водопадный, водопадный и закритический;
	
	--- результаты исследования эффективности показали, что рациональным режимом работы мельницы является каскадно-водопадный в силу наибольшего роста скорости изменения эффективности перемалывания, а также на практике необходимо исходить из прочностных характеристик мельницы и энергетических затрат производства;
	
	--- для полноценного исследования эффективности режимов работы также следует рассматривать скорость подачи руды в качестве дополнительного параметра режима.


\section{Шар-стенка}

1) Стенка представлена как замкнутая фигура, состоящая из конечного числа прямых линий.

2) Шар никак не влияет на стенку. 
Её движение зависит только от заданного ей закона движения.

3) При расчёте сил трения используется не эффективный радиус, а радиус данного шара.

4) Стенка вращается относительно какой-то точки и зацеплении шар-стена рассматривается как внутреннее, а не внешнее. 
Это изменит знак угловой скорости стенки при расчете относительной угловой скорости шара.


\section{Упрощения МДЭ}

В методе дискретных элементов есть два основных упрощения.

1) Выбранный временной шаг настолько мал, что в течение одного временного шага возмущения не могут распространяться с любого элемента дальше, чем на его ближайших соседей. 

2) Считается, что шары не деформируются.
Деформации отдельных частиц малы по сравнению с изменением объёма дискретной среды в целом.
А потому шары просто накладываются друг на друга.

\section{Силы трения скольжения}

Так как сила трения скольжения появится в точке контакта, то при перемещении ее в центр появляется момент от данной силы.
На данном слайде представлен механизм переноса силы в центр тяжести.

\end{document}