\documentclass[a4paper]{article}
\usepackage[14pt]{extsizes}
%%% Работа с русским языком
\usepackage{cmap}					% поиск в PDF
\usepackage{mathtext} 				% русские буквы в фомулах
\usepackage[T2A]{fontenc}			% кодировка
\usepackage[utf8]{inputenc}			% кодировка исходного текста
\usepackage[english,russian]{babel}	% локализация и переносы

%%% Дополнительная работа с математикой
\usepackage{amsfonts,amssymb,amsthm,mathtools} % AMS
\usepackage{amsmath}
\usepackage{icomma} % "Умная" запятая: $0,2$ --- число, $0, 2$ --- перечисление

%% Номера формул
%\mathtoolsset{showonlyrefs=true} % Показывать номера только у тех формул, на которые есть \eqref{} в тексте.

%% Шрифты
\usepackage{euscript}	 % Шрифт Евклид
\usepackage{mathrsfs} % Красивый матшрифт

%% Свои команды
\DeclareMathOperator{\sgn}{\mathop{sgn}}

%% Перенос знаков в формулах (по Львовскому)
\newcommand*{\hm}[1]{#1\nobreak\discretionary{}
{\hbox{$\mathsurround=0pt #1$}}{}}

%%% Работа с картинками
\usepackage{float}
\usepackage{graphicx}  % Для вставки рисунков
\graphicspath{{pictures/}}  % папки с картинками
\setlength\fboxsep{3pt} % Отступ рамки \fbox{} от рисунка
\setlength\fboxrule{1pt} % Толщина линий рамки \fbox{}
\usepackage{wrapfig} % Обтекание рисунков и таблиц текстом

%%% Работа с таблицами
\usepackage{array,tabularx,tabulary,booktabs} % Дополнительная работа с таблицами
\usepackage{longtable}  % Длинные таблицы
\usepackage{multirow} % Слияние строк в таблице

\usepackage{cite}
\usepackage{csquotes}


%%% Заголовок
\author{Катнов Артем}
\title{Курсовая работа по теме: \\ Численное моделирование динамики частиц дроби в рудоразмольной мельнице методом дискретных элементов}
\date{\today}

\usepackage[left=3cm,right=1.5cm,top=2cm,bottom=2cm]{geometry}
\linespread{1.5}
%\parindent=1.25cm
%\oddsidemargin=4.6mm
%\textwidth=16cm
%\headheight=0cm
%\headsep=0cm
%\topmargin=-1cm
%\textheight=25.7cm

%\usepackage{caption}
%\usepackage{flafter}
%\usepackage{footmisc}

\usepackage{xcolor}
\usepackage{hyperref}
% цвета для гиперссылок
\definecolor{linkcolor}{HTML}{0000FF} % цвет ссылок
\definecolor{urlcolor}{HTML}{0000FF} % цвет гиперссылок
\hypersetup{linkcolor=linkcolor, urlcolor=urlcolor, colorlinks=true}

\usepackage{cleveref}
%\usepackage{underscore}
%\usepackage{etoolbox}
%\usepackage{lastpage}
%\usepackage{titlesec}
%\usepackage{flafter}
%\usepackage{color}
%\usepackage{mfirstuc}
%\usepackage{nomencl}
%\usepackage{iftex}

\begin{document}

\section{Титульник}

Добрый вечер уважаемая комиссия, меня зовут Артем Катнов, мой научный руководитель Жуков Никита Александрович, тема моей курсовой работы "Численное моделирование динамики частиц дроби в рудоразмольной мельнице методом дискретных элементов".

\section{Содержание}

В данной презентации я бы хотел рассказать об идее метода дискретных элементов, описать построенную математическую модель и продемонстрировать в рамках данной модели рудоразмольную мельницу при различных режимах работы.

\section{Дискретная среда}

Метод дискретного элемента -- это семейство численных методов предназначенных для расчёта движения большого количества частиц, таких как молекулы, песчинки, гравий и прочих гранулированных сред.
Метод был первоначально применён Cundall в 1971 для решения задач механики горных пород.
На слайде представлены различные среды, которые могут считаться дискретными.

\section{Цель работы}

Барабанная мельница (рисунок \ref{pic:baraban_shema}) представляет собой пустотелый барабан, который вращается вокруг горизонтальной оси.
При его вращении дробящие тела  благодаря трению увлекаются его внутренней поверхностью, поднимаются на некоторую высоту и свободно или перекатываясь падают вниз.

Задача данной работы: применение метода дискретных элементов для моделирования динамики частиц дроби во вращающейся мельницы. 
Это поможет исследовать работу мельницы при различных режимах.

\section{Алгоритм в общем виде}

Моделирование динамики системы происходит во времени.
Данная модель предполагает наличие всего трех степеней свободы.
На каждом шаге при нахождении пересекающихся шаров для них происходит переход в локальную систему координат и расчёт всех возникающих сил, потом перевод всех найденных значений в глобальную систему.
В зависимости от сил, действующих на частицу, происходит выбор закона движения на данном шаге.
При наличии контактных сил проводится итерационная процедура ввиду нелинейности решаемой задачи.
Моделирование пересечения происходит и между частицами непосредственно и между частицами и стенками шаровой мельницы.

В общем виде алгоритм метода представляет собой цикл в котором происходит сначала расчет всех сил действущих на каждую из частиц, а потом обновление положения частиц по принятому кинематическому закону с учетом найденных сил.



\section{Контактные силы}
На данном слайде представлены все контактные силы действующие на шары в момент взаимодействия.
Это нормальная отталкивающая сила, сила трения скольжения и момент трения качения.
Подобное моделирование силовых факторов от трения описано в статье, представленной на слайде и довольно распространено при построении МДЭ.

\section{Контактные силы в нормальном направлении}

Контактные силы находятся из решения контактной задачи Герца.
Таким образом жесткость считается по представленной на слайде формуле.
Такой подход позволяет учесть и свойства шаров и размер, и то как сильно они вошли друг в друга на данном этапе.

\section{Контактные силы в тангенциальном и окружном направлениях}

На данном слайде представлены формулы по которым (в рамках задачи Герца) находятся силовые факторы в тангенциальном и окружном направлениях. 
Соответственно они зависят от направления относительных скоростей, значения соответствующих коэффициентов трения и нормальной силы.

\section{Силы диссипации}

Взаимодействия двух шаров можно представить в виде пружины и демпфера.
Если пружину мы уже обсудили, то демпфер нет.
Он порождает силы диссипации, которые зависят от скорости, размеров и свойств шаров.
Находятся силы диссипации тоже из решения контактной задачи Герца и рассчитываются в нормальном и тангенциальном направлениях.

\section{Кинематика}

Думаю важно сделать оговорку, что в этой работе происходит расчёт для трёх степеней свободы.
Задача которую решает данный метод изначально нелинейна, потому в расчет кинематики частицы на шаге вводится так называемый рывок.
Его мы находим в ходе итерационного уточнения.
На слайде он обозначен переменной $b$ с соответствующим индексом.

\section{Итерационное уточнение}

Итерационное уточнения представляет собой цикл, целью которого является нахождение ускорения в следующий момент времени.
После нахождения данного ускорения мы и определяем рывок по формулам, представленным на слайде.
Расчёт рывка происходит по каждому их трёх стрепеней свободы.

\section{Алгоритм итерационного уточнения}

На данном слайде представлен алгоритм итерационного уточнения.
В ходе этого цикла мы предполагаем следующее положение шара (исходя из его скорости и ускорения от данного взаимодействия) и по закону расчёта контактных сил находим обновленную силу и соответственно усккорение.
Расчет продолжается пока изменение ускорения в ходе итерационного процесса нельзя будет считать достаточно малым.

\section{Результаты работы: постановка задачи}

В данной работе решалась задача моделирования только частиц дроби в рудоразмольной мельнице.
И дробь и сама мельница брались как выполненные из металла.
Соотвественно все физические харктеристики здесь приведены.
Мельница заполнялась на 21 процент и временной маг был выбран как 10$^{-5}$ секунды.

\section{Результаты работы: каскадный режим}

При работе шаровой мельницы можно выделить несколько режимов, которые зависят от угловой скорости вращения стенки.
Рассмотрим их последовательно.

Каскадный режим достигается при примерно 2 оборотах в минуту и выделяется перекатыванием шаров друг по другу.

\section{Результаты работы: смешанный режим}

При смешанном режиме работе шаровой мельницы шары отрываются от стены, но не перелетают, а падают на другие шары. Достигается при 14 обортах в минуту.

\section{Результаты работы: водопадный режим}

Водопадный режим достигается при примерно 17 оборотах в минуту и выделяется отрыванием шаров на определенной высоте и перелетом через остальные шары по параболе.

\section{Результаты работы: с превышением критической частоты}

При значительном превышении критической скорости можем наблюдать почти мгновенный эффект приклеивания шаров к стенам.

\section{Вывод}

На этом у меня все.
В представленной работе изучен метод дискретных элементов, а также была разработана математическая модель динамики частиц дроби в шаровой рудоразмольной мельнице.
В качестве примера применимости и работоспособности данной математической модели представлено сравнение с изученной моделью шаровой мельницы.
Разработанная модель позволяет исследовать режимы работы рудоразмольной мельницы.
В дальнейшем планируется добавить частицы руды и модель разрушения для частиц руды.


\section{Шар-стенка}

1) Стенка представлена как замкнутая фигура, состоящая из конечного числа прямых линий.

2) Шар никак не влияет на стенку. 
Её движение зависит только от заданного ей закона движения.

3) При расчёте сил трения используется не эффективный радиус, а радиус данного шара.

4) Стенка вращается относительно какой-то точки и зацеплении шар-стена рассматривается как внутреннее, а не внешнее. 
Это изменит знак угловой скорости стенки при расчете относительной угловой скорости шара.


\section{Упрощения МДЭ}

В методе дискретных элементов есть два основных упрощения.

1) Выбранный временной шаг настолько мал, что в течение одного временного шага возмущения не могут распространяться с любого элемента дальше, чем на его ближайших соседей. 

2) Считается, что шары не деформируются.
Деформации отдельных частиц малы по сравнению с изменением объёма дискретной среды в целом.
А потому шары просто накладываются друг на друга.

\end{document}