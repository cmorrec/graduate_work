\documentclass[a4paper]{article}
\usepackage[14pt]{extsizes}
%%% Работа с русским языком
\usepackage{cmap}					% поиск в PDF
\usepackage{mathtext} 				% русские буквы в фомулах
\usepackage[T2A]{fontenc}			% кодировка
\usepackage[utf8]{inputenc}			% кодировка исходного текста
\usepackage[english,russian]{babel}	% локализация и переносы

%%% Дополнительная работа с математикой
\usepackage{amsfonts,amssymb,amsthm,mathtools} % AMS
\usepackage{amsmath}
\usepackage{icomma} % "Умная" запятая: $0,2$ --- число, $0, 2$ --- перечисление

%% Номера формул
%\mathtoolsset{showonlyrefs=true} % Показывать номера только у тех формул, на которые есть \eqref{} в тексте.

%% Шрифты
\usepackage{euscript}	 % Шрифт Евклид
\usepackage{mathrsfs} % Красивый матшрифт

%% Свои команды
\DeclareMathOperator{\sgn}{\mathop{sgn}}

%% Перенос знаков в формулах (по Львовскому)
\newcommand*{\hm}[1]{#1\nobreak\discretionary{}
{\hbox{$\mathsurround=0pt #1$}}{}}

%%% Работа с картинками
\usepackage{float}
\usepackage{graphicx}  % Для вставки рисунков
\graphicspath{{pictures/}}  % папки с картинками
\setlength\fboxsep{3pt} % Отступ рамки \fbox{} от рисунка
\setlength\fboxrule{1pt} % Толщина линий рамки \fbox{}
\usepackage{wrapfig} % Обтекание рисунков и таблиц текстом

%%% Работа с таблицами
\usepackage{array,tabularx,tabulary,booktabs} % Дополнительная работа с таблицами
\usepackage{longtable}  % Длинные таблицы
\usepackage{multirow} % Слияние строк в таблице

\usepackage{cite}
\usepackage{csquotes}


%%% Заголовок
\author{Катнов Артем}
\title{Курсовая работа по теме: \\ Численное моделирование динамики частиц дроби в рудоразмольной мельнице методом дискретных элементов}
\date{\today}

\usepackage[left=3cm,right=1.5cm,top=2cm,bottom=2cm]{geometry}
\linespread{1.5}
%\parindent=1.25cm
%\oddsidemargin=4.6mm
%\textwidth=16cm
%\headheight=0cm
%\headsep=0cm
%\topmargin=-1cm
%\textheight=25.7cm

%\usepackage{caption}
%\usepackage{flafter}
%\usepackage{footmisc}

\usepackage{xcolor}
\usepackage{hyperref}
% цвета для гиперссылок
\definecolor{linkcolor}{HTML}{0000FF} % цвет ссылок
\definecolor{urlcolor}{HTML}{0000FF} % цвет гиперссылок
\hypersetup{linkcolor=linkcolor, urlcolor=urlcolor, colorlinks=true}

\usepackage{cleveref}
%\usepackage{underscore}
%\usepackage{etoolbox}
%\usepackage{lastpage}
%\usepackage{titlesec}
%\usepackage{flafter}
%\usepackage{color}
%\usepackage{mfirstuc}
%\usepackage{nomencl}
%\usepackage{iftex}

\begin{document}

\section{Титульник}

Добрый вечер уважаемая комиссия, меня зовут Артем Катнов, мой научный руководитель Жуков Никита Александрович, тема моей исследовательской работы "Численное моделирование динамики частиц в рудоразмольной мельнице методом дискретных элементов".

\section{Содержание}

В данной презентации я бы хотел рассказать об идее метода дискретных элементов, описать построенную математическую модель и обсудить применение данного метода к конкретной задаче рудоразмольной шаровой мельницы.

\section{Дискретная среда}

Метод дискретного элемента -- это семейство численных методов предназначенных для расчёта движения большого количества частиц, таких как молекулы, песчинки, гравий и прочих гранулированных сред.
Метод был первоначально применён Cundall в 1971 для решения задач механики горных пород.
На слайде представлены различные среды, которые могут считаться дискретными.

\section{Цель работы}

Барабанная мельница представляет собой пустотелый барабан, который вращается вокруг горизонтальной оси.
При его вращении дробящие тела  благодаря трению увлекаются его внутренней поверхностью, поднимаются на некоторую высоту и свободно или перекатываясь падают вниз.

Задача данной работы: применение метода дискретных элементов для моделирования динамики частиц во вращающейся мельнице. 
Это поможет исследовать работу мельницы при различных режимах.

\section{Алгоритм в общем виде}

Моделирование динамики системы происходит во времени.

В общем виде алгоритм метода представляет собой цикл в котором происходит сначала расчет всех сил действущих на каждую из частиц, а потом обновление положения частиц по принятому кинематическому закону с учетом найденных сил.
В ходе расчета сил также проводится итерационная процедура (в виду нелинейности задачи).
В ходе же обновления частиц также их количество может поменяться (если какая-то из частиц руды будет разрушена).



\section{Алгоритм в частном виде}
Данная модель предполагает наличие всего трех степеней свободы.
На слайде продемонстрированы три возможных состояния частицы (возможны также различные комбинации первых двух):

--- зацепление частица-частица;

--- зацепление частица-шар;

--- частица в свободном полете.

На каждом шаге при нахождении пересекающихся шаров для них происходит переход в локальную систему координат и расчёт всех возникающих сил, потом перевод всех найденных значений в глобальную систему.
В зависимости от сил, действующих на частицу, происходит выбор закона движения на данном шаге.
При наличии контактных сил проводится итерационная процедура ввиду нелинейности решаемой задачи.


\section{Контактные силы}
На данном слайде представлены все контактные силы действующие на шары в момент взаимодействия.
Это нормальная отталкивающая сила, сила трения скольжения и моменты трения качения и скольжения.
Подобное моделирование силовых факторов от трения описано в статье, представленной на слайде и довольно распространено при построении МДЭ.

\section{Контактные силы в нормальном направлении}

Контактные силы находятся из решения контактной задачи Герца.
Таким образом жесткость считается по представленной на слайде формулеи сила зависит от вхождения нелинейно.
Такой подход позволяет учесть и свойства шаров и размер, и то как сильно они вошли друг в друга на данном этапе.

\section{Контактные силы в тангенциальном и окружном направлениях}

На данном слайде представлены формулы по которым (в рамках задачи Герца) находятся силовые факторы в тангенциальном и окружном направлениях. 
Соответственно они зависят от направления относительных скоростей, значения соответствующих коэффициентов трения и нормальной силы.

\section{Силы трения скольжения}

Так как сила трения скольжения появится в точке контакта, то при перемещении ее в центр появляется момент от данной силы.
На данном слайде представлен механизм переноса силы в центр тяжести.

\section{Силы диссипации}

Взаимодействия двух шаров можно представить в виде пружины и демпфера.
Если пружину мы уже обсудили, то демпфер нет.
Он порождает силы диссипации, которые зависят от скорости, размеров и свойств шаров.
Находятся силы диссипации тоже из решения контактной задачи Герца и рассчитываются в нормальном и тангенциальном направлениях.

\section{Кинематика}

Думаю важно сделать оговорку, что в этой работе происходит расчёт для трёх степеней свободы.
Задача которую решает данный метод изначально нелинейна, потому в расчет кинематики частицы на шаге вводится так называемый рывок.
Его мы находим в ходе итерационного уточнения.
На слайде он обозначен переменной $b$ с соответствующим индексом.

Итерационное уточнение представляет собой цикл, целью которого является нахождение ускорения в следующий момент времени.
После нахождения данного ускорения мы и определяем рывок по формулам, представленным на слайде.
Расчёт рывка происходит по каждому их трёх стрепеней свободы.

После нахождения рывков и ускорений в локальной системе координат (в нормальном и тангенциальном направлениях) происходит переход в глобальную систему координат через матрицу перехода.

\section{Алгоритм итерационного уточнения}

На данном слайде представлен алгоритм итерационного уточнения.
В ходе этого цикла мы предполагаем следующее положение шара (исходя из его скорости и ускорения от данного взаимодействия) и по закону расчёта контактных сил находим обновленную силу и соответственно ускорение.
Расчет продолжается пока изменение ускорения в ходе итерационного процесса нельзя будет считать достаточно малым.

\section{Совокупность уравнений}

В случае контакта частицы с чем-либо (шаром или стенкой) совокупность уравнений выглядит следующим образом: ускорение зависит от контактных сил, сил диссипации и силы поля (в нашем случае это сила гравитации).
После нахождения ускорения происходит итерационная уточняющая процедура.
Далее мы можем найти скорость на данном шаге и положение элемента в пространстве путем простого интегрирования.
Это показано векторно для линейных степеней свободы и аналогично для угловой степени свободы.

\section{Совокупность уравнений: без контакта}

В случае отсутствия контакта единственная сила, действующая на частицу -- сила поля.
На вращение же элемента не влияет вообще ничего.
Итерационная процедура не проводится и рывок принимается равным нулю.
В таком случае уравнения будут выглядеть следующим образом.

\section{Модель разрушения}

Для расчета используется модель замещения баланса популяции (PBRM) - частицы заменяются набором дочерних фрагментов в момент разрушения. 
Как это выглядит будет показано на следующем слайде а здесь я бы хотел обсудить в каком случае частица будет считаться разрушенной.
Перед вами представлен энергетический метод.
Каждый шар хранит накопленную энергию разрушения.
Определяется она по формуле \_ .
Добавок энергии Е определяется энергией сжатой пружины для частицы.
Далее происходит расчет вероятности разлома частицы и определение того сломается она или нет с заданной вероятностью.
Емин и С эмпирические коэффициенты, получаемые для каждого материала и набора данных в ходе эксперимента.

\section{Модель разрушения: проблема}

Далее, если частица разрушена, то с помощью нормального распределения случайным образом определяется количество новых частиц, образованных на месте старой (с большой долей вероятности это значение будет от 2 до 7).
Расчет размера и положения новых частиц происходит из равенства масс и кинетической энергии.
С этим могут быть сложности при работе в двухмерной постановке.
Расчет размера новых шаров производится по представленным формулам (конкретно здесь для двух шаров), но т.к. они отображаются в двухмерном пространстве, то площади их считаются как представлено ниже.
Т.е. площадь занимаемого пространства зрительно закономерно увеличивается при появлении новых частиц.
На рисунке \_ представлено появлении новых двух шаров на месте старого.

Далее 3 и 4 соотвественно.


\section{Рудоразмольная мельница}

Вся эта модель построена для расчета различных режимов работы шаровой мельницы.
На слайде представлено в общем виде что она из себя представляет.
Через одну из цапф происходит загрузка материала, внутри барабана происходит перемалывание, и через другую цапфу происходит выгрузка материала, который может пройти через сито.
Для обеспечения подобного эффекта в данной работе периодически происходит добавление новой частицы руды в предположительно наименее забитую шарами часть мельницы и также удаление частиц руды, проходящих через сито.
Сито в свою очередь конечно имеет некоторую пропускную способность и туда проходят только достаточно маленькие частицы руды.
Расположено оно по центру аналогично рисунку.


\section{Реальные параметры}

На данном слайде представлены реальные параметры величин, учавствующих в расчетах.
Здесь представлены величины имеющие реальные параметры, а на следующем слайде величины полученные эмпирически и в целом необходимые в доработке для конкретных значений.
  


\section{Результаты}

\subsection{Каскадный режим}

характеризуется перекатыванием шаров по другим слоям

\subsection{Смешанный режим}

\subsection{Водопадный режим}


\subsection{Закритический режим}

\subsection{График распределения эффективности работы мельницы}

На графике представлена зависимость эффективности перемалывания руды от скорости вращения шаровой мельницы. 
Эффективность определяется как отношение объема просеянной руды к суммарному объему руды.
Как видно из этого графика при увеличении скорости растет и эффективность работы мельницы.
Стоит отметить, что для всех режимов работы выбрана одинаковая скорость подачи руды.
В какой-то момент выбранная скорость подачи материала в цапфу становится недостаточна.

Также здесь явно прослеживается выход графика на горизонтальную асимптоту -- скорость приращения эффективности после точки 7.5 рад / с падает. 
Таким образом дальнейшее увеличение скорости вращения мельницы не приводит к значительному увеличению эффективности ее работы. В то же время -- практике же используют каскадно-водопадный и водопадный режимы работы.
Это связано с прочностными характеристиками мельницы и энергетическими затратами по содержанию и обслуживанию мельницы.
Таким образом результаты работы модели подтверждат, что выбор каскадно-водопадного режима является рациональным с точки зрения эффективности мельницы и накладываемых прочностных и экономических ограничений

Рост эффективности (вплоть до 1) при больших скоростях вращения барабана связан с тем, что скорость разрушения и вывода руды из полости барабана превышает скорость подачи.
Логично предположить, что при увеличении скорости вращения мельницы также должна увеличиваться частота подачи материала.


\section{Вывод}

На этом у меня все.
В представленной работе изучен метод дискретных элементов, а также была разработана математическая модель динамики частиц дроби в шаровой рудоразмольной мельнице.
В качестве примера применимости и работоспособности данной математической модели представлено сравнение с изученной моделью шаровой мельницы.
Разработанная модель позволяет исследовать режимы работы рудоразмольной мельницы.

Также в данной работе:

	--- рассмотрены 4 характерных режима работа мельницы: каскадный, каскадно-водопадный, водопадный и закритический;

	--- произведена оценка эффективности для каждого из рассмотренных режимов;

	--- результаты исследования эффективности показали, что рациональным режимом работы мельницы является каскадно-водопадный в силу наибольшего роста скорости изменения эффективности перемалывания, а также на практике необходимо исходить из прочностных характеристик мельницы и энергетических затрат производства;
	
	--- для полноценного исследования эффективности режимов работы также следует рассматривать скорость подачи руды в качестве дополнительного параметра режима.


\section{Шар-стенка}

1) Стенка представлена как замкнутая фигура, состоящая из конечного числа прямых линий.

2) Шар никак не влияет на стенку. 
Её движение зависит только от заданного ей закона движения.

3) При расчёте сил трения используется не эффективный радиус, а радиус данного шара.

4) Стенка вращается относительно какой-то точки и зацеплении шар-стена рассматривается как внутреннее, а не внешнее. 
Это изменит знак угловой скорости стенки при расчете относительной угловой скорости шара.


\section{Упрощения МДЭ}

В методе дискретных элементов есть два основных упрощения.

1) Выбранный временной шаг настолько мал, что в течение одного временного шага возмущения не могут распространяться с любого элемента дальше, чем на его ближайших соседей. 

2) Считается, что шары не деформируются.
Деформации отдельных частиц малы по сравнению с изменением объёма дискретной среды в целом.
А потому шары просто накладываются друг на друга.


\end{document}